\documentclass{article}
\usepackage[utf8]{inputenc}
\usepackage[acronym, toc]{glossaries}

\makeglossaries

\newglossaryentry{first}{
	name={first},
	description={first}
}



\begin{document}

\tableofcontents

\section{First Section}

The \Gls{latex} typesetting markup language is specially suitable for documents that include \gls{maths}. \Glspl{formula} are rendered properly an easily once one gets used to the commands.


\clearpage

\section{Second Section}

\vspace{5mm}

Given a set of numbers, there are elementary methods to compute its \acrlong{gcd}, which is abbreviated \acrshort{gcd}. This process is similar to that used for the \acrfull{lcm}.


\clearpage

\printglossary

\clearpage

\printglossary[type=\acronymtype]

\end{document}
