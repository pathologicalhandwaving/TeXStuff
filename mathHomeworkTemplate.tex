\documentclass[10pt]{article}

% Document Information
\author{}
\title{}
\date{\today}

% Basic Packages
\usepackage{fullpage}
\usepackage{hyperref}
\usepackage{float}

\usepackage[parfill]{parskip}
\usepackage{tikz}
\usepackage{amsmath}
\usepackage{amsthm}
\usepackage{amssymb}
\usepackage{stmaryrd}

% A "fancy" package
\usepackage{fancyhdr}
\pagestyle{fancy}
\usepackage{lastpage}
\setlength{\headheight}{25.2pt}
\setlength{\headsep}{0.4in}
\pagestyle{fancy}
\fancyhf{}
\rhead{\textbf{Class Name}}
\lhead{\textbf{\author}}
\rfoot{Page \thepage/\pageref{LastPage}}

% Envs made up in theorem, definition, and remark style (most not usually used)
\theoremstyle{plain}
\newtheorem{thm}{Theorem}
\newtheorem{lem}{Lemma}
\newtheorem{cor}{Corollary}
\newtheorem{prop}{Proposition}
\newtheorem{con}{Conjecture}
\newtheorem{crit}{Criterion}
\newtheorem{ass}{Assertion}

\theoremstyle{definition}
\newtheorem{defn}{Definition}
\newtheorem{exmp}{Example}
\newtheorem{xca}[exmp]{Exercise}
\newtheorem{cond}{Condition}
\newtheorem{prob}{Problem}
\newtheorem*{sol}{Solution}
\newtheorem*{asl}{Alternate Solution}
\newtheorem{algo}{Algorithm}
\newtheorem{que}{Question}
\newtheorem{ans}{Answer}
\newtheorem{axi}{Axiom}
\newtheorem{prot}{Property}
\newtheorem{asu}{Assumption}
\newtheorem{hyp}{Hypothesis}

\theoremstyle{remark}
\newtheorem{rem}{Remark}
\newtheorem*{nt}{Note}
\newtheorem*{nota}{Notation}
\newtheorem{cla}{Claim}
\newtheorem{summ}{Summary}
\newtheorem{case}{Case}
\newtheorem{cln}{Conclusion}

% one problem per page

%%%%%%%%%%%%%%%%%%%%%%%%%%%%%%%%%%%%
\begin{document}

\begin{prob}

\end{prob}

\medskip

\begin{sol}

\end{sol}

\pagebreak

\begin{prob}

\end{prob}

\medskip

\begin{sol}

\end{sol}

\medskip

% Roman Numeral List
\begin{enumerate}[label=(\roman*)]
    \item 
    \item 
\end{enumerate}

\pagebreak

\begin{prob}

\end{prob}

\medskip

\begin{sol}

\end{sol}

\medskip

\begin{description}
    \item 
    \item 
\end{description}

\pagebreak

\begin{prob}

\end{prob}

\medskip

% Proof
\begin{proof}

\end{proof}

\pagebreak

\begin{prob}

\end{prob}

\medskip

\begin{sol}

\end{sol}

\medskip

\begin{figure}[H]   %use h for place here if possible, use H for place here no matter what
\begin{center}
\includegraphics[scale=0.50]{ REPLACE WITH PATH} 
\end{center}
\caption{ REPLACE WITH CAPTION }
\end{figure}


\pagebreak

\begin{prob}

\end{prob}

\medskip

\begin{sol}

\end{sol}

\medskip

\begin{table}
\begin{center}
\begin{tabular}{|c|c|c|c|c|c|}
\hline 
• & (-6) & (-3) & 0 & 3 & 6 \\ 
\hline 
(-6) & -12 & -9 & -6 & -3 & 0 \\ 
\hline 
(-3) & -9 & -6 & -3 & 0 & 3 \\ 
\hline 
0 & -6 & -3 & 0 & 3 & 6 \\ 
\hline 
3 & -3 & 0 & 3 & 6 & 9 \\ 
\hline 
6 & 0 & 3 & 6 & 9 & 12 \\ 
\hline 
\end{tabular} 
\caption{This table is for the first condition of $i + j$, for $i$ rows and $j$ columns.}
\end{center}
\end{table}

\pagebreak

\begin{prob}

\end{prob}

\medskip

\begin{sol}

\end{sol}

\medskip

\begin{tikzpicture}

\end{tikzpicture}

\end{document}
