\documentclass[10pt]{article}

\usepackage[parfill]{parskip}
\usepackage{xcolor}
\usepackage{amsmath}
\usepackage{verbatim}
\usepackage[none]{hyphenat}


\newcommand{\typer}[1]{\textbf{Typesetting Remark:} #1}
\newcommand{\stl}[1]{\textbf{Wording Remark:} #1} 
\newcommand{\omt}[1]{\textbf{Omit:} #1}
\newcommand{\punt}[1]{\textbf{Punctuation:} #1}
\newcommand{\cla}[1]{\textbf{Clarification Needed:} #1}

\newcommand{\obj}[1]{\textbf{\color{red}{Goal of Paper:}} #1}
\newcommand{\sys}[1]{\textbf{\color{red}{Goal of Application:}} #1}

\newcommand{\terr}[1]{\textbf{Technical Error:} #1}
\newcommand{\trem}[1]{\textbf{Technical Remark:} #1}




%%%%%%%%%%%%%%%%%%%%%%%%%%%%%%%%%%%%%%%%%%%%
\begin{document}

\section*{Draft Review: N. Kinkel}
\textbf{Date:} January 7, 2017 \\
\textbf{Version of Review:} 1.0 \\
\medskip

\subsection*{Reading Instructions}

The reviews of material are given in coordinate form: $(line, \ word)$ \\
This enables easy location of the material to which the review is referring. Note that for this reason the review is specific to the version of the draft. 

The references given at the end of the review were used as reference material during the analysis of the draft and may not completely exhaust all resources used by the reviewer in the technical portions. That is, a lot of this shit is already in my head, it is easily verifiable usually by a quick Google or Wikipedia search. Deal with it.

\subsection*{Specific Types of Remarks}

\begin{description}
    \item [Typesetting Remark] Refers to a suggested latex or stylistic change in the content.
    \item [Wording Remark] Refers to a suggested change in wording, usually followed by a suggested way of changing the content
    \item [Omit] Refers to a suggested change in a single word, phrase, sentence, paragraph, or section
    \item [Punctuation] Self-explanitory
    \item [Clarification Needed] Refers to some object in the context that requires some form of clarification before the reviewer may pass judgement on that portion of the text
    \item [Goal of Paper] Content noted to aid the reviewer not necessarily the author
    \item [Goal of Application]  Content noted to aid the reviewer not necessarily the author
    \item [Technical Error] Technical error or possible technical error
    \item [Technical Remark] Self-explanatory but the author should probably indicate in the next draft they understand the purpose of said remarks inclusion in the review unless they would not like another review.
\end{description}
\medskip


\bigskip
\hrule
\bigskip

%%%%%
\section*{Review}






\bibliographystyle{alpha}
\nocite{*}
\bibliography

\end{document}