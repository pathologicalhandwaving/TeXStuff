%%%%%%%%%%%%%%%%%%%%%%%%%%%%%%%%%%%%%%%%%
% Article Notes
% LaTeX Template
% Version 1.0 (1/10/15)
%
% This template has been downloaded from:
% http://www.LaTeXTemplates.com
%
% Authors:
% Vel (vel@latextemplates.com)
% Christopher Eliot (christopher.eliot@hofstra.edu)
% Anthony Dardis (anthony.dardis@hofstra.edu)
%
% License:
% CC BY-NC-SA 3.0 (http://creativecommons.org/licenses/by-nc-sa/3.0/)
%
%%%%%%%%%%%%%%%%%%%%%%%%%%%%%%%%%%%%%%%%%

%----------------------------------------------------------------------------
%	PACKAGES AND OTHER DOCUMENT CONFIGURATIONS
%----------------------------------------------------------------------------

\documentclass[10pt,letterpaper,twocolumn,landscape]{article}

%%%%%%%%%%%%%%%%%%%%%%%%%%%%%%%%%%%%%%%%%
% Math Article Notes
% Structure Specification File
% Version 2.5 (04/29/2016)
%
% This file was modified from a template 
% which was downloaded from:
% http://www.LaTeXTemplates.com
%
% Original Authors:
% Vel 
% (vel@latextemplates.com)
% Christopher Eliot 
% (christopher.eliot@hofstra.edu)
% Anthony Dardis 
% (anthony.dardis@hofstra.edu)
%
% License:
% CC BY-NC-SA 3.0 
% (http://creativecommons.org/licenses/by-nc-sa/3.0/)
%
% Current Version Author:
% Kris
% (kshort@iastate.edu)
% (https://www.github.com/krismshort)
%
%%%%%%%%%%%%%%%%%%%%%%%%%%%%%%%%%%%%%%%%%

%----------------------------------------------------------------------------------------
%	REQUIRED PACKAGES
%----------------------------------------------------------------------------------------
\usepackage[columnsep=1cm, left=0.5in, right=0.5in, margin=0.25in,bottom=.05in,top=.15in, bindingoffset=0.2in]{geometry}
\setlength{\voffset}{-0.75in}
\setlength{\headsep}{0pt}


\usepackage{hyperref}
\usepackage{url}

\usepackage{amsmath}
\usepackage{amssymb}
\usepackage{amsthm}
\usepackage{fullpage}
\usepackage{hyperref}
\usepackage{float}
\usepackage{stmaryrd}


\usepackage{amsrefs}
\bibliographystyle{amsalpha}
\usepackage{gloss}

\usepackage[english]{babel} % Use english by default

\usepackage{todonotes}
\usepackage{concepts}

\usepackage{enumerate}

% AMS Theorem Package Custom Definitions

\theoremstyle{definition}
\newtheorem*{defn}{Definition. \\}
\newtheorem{que}{Question: \\}
%\newtheorem{ansfd}{Answer Found:}

\theoremstyle{theorem}
\newtheorem*{thm}{Theorem}{}

\theoremstyle{remark}
\newtheorem{rem}{Remark: \\}
\newtheorem*nt}{Note: \\}
\newtheorem{ans}{Answer: \\}


% Ways to number lists

%\renewcommand{\labelenumi}{\Roman{enumi}}
%\renewcommand{\labelitemii}{$\circ$}}
%\renewcommand{\labelitemiii}{$\bigstar$}
%\renewcommand{\labelitemiv}{$\centerdot$}
%\renewcommand{\labelitemi}{$\bullet$}
%\renewcommand{\labelitemii}{$\cdot$}
%\renewcommand{\labelitemiii}{$\diamond$}
%\renewcommand{\labelitemiv}{$\ast$}
%\renewcommand{\labelitemiv}{$\ast$}
%\alph	Lowercase letter (a, b, c, ...)
%\Alph	Uppercase letter (A, B, C, ...)
%\arabic	Arabic number (1, 2, 3, ...)
%\roman	Lowercase Roman numeral (i, ii, iii, ...)
%\Roman	Uppercase Roman numeral (I, II, III, ...)

% change separation between items in lists
% DOES NOT SEEM TO WANT TO WORK
%\begin{itemize}\itemsep2pt

\usepackage{verbatim}
%----------------------------------------------------------------------------------------
%	CUSTOM COMMANDS
%----------------------------------------------------------------------------------------
% Question Information

% Answer Information
\newcommand{\answerfound}[1]{\renewcommand{\answerfound}{#1}}
\newcommand{\dateanswerfound}[1]{\renewcommand{\dateanswerfound}{#1}}
\newcommand{\answerauthor}[1]{\renewcommand{\answerauthor}{#1}}
% Template Revision Number
\newcommand{\revisionnumber}[1]{\renewcommand{\revisionnumber}{#1}}

% Article Information
\newcommand{\articletitle}[1]{\renewcommand{\articletitle}{#1}} % Define a command for storing the article title
\newcommand{\articlecitation}[1]{\renewcommand{\articlecitation}{#1}} % Define a command for storing the article citation
\newcommand{\ckey}[1]{\renewcommand{\ckey}{#1}} % Plain format cite key
\newcommand{\doctitle}{\articlecitation\ --- ``\articletitle''} % Define a command to store the article information as it will appear in the title and header

\newcommand{\datenotesstarted}[1]{\renewcommand{\datenotesstarted}{#1}} % Define a command to store the date when notes were first made
\newcommand{\docdate}[1]{\renewcommand{\docdate}{#1}} % Define a command to store the date line in the title
\newcommand{\docauthor}[1]{\renewcommand{\docauthor}{#1}} % Define a command for storing the article notes author
\newcommand{\amsclass}[1]{\renewcommand{\amsclass}{#1}} % AMS Classification
\newcommand{\amsreview}[1]{\renewcommand{\amsreview}{#1}} % AMS Review

% Header and Footer
\usepackage{fancyhdr}
\pagestyle{fancy}
\usepackage{lastpage}

\setlength{\headheight}{25.2pt}
\setlength{\headsep}{0.4in}
\renewcommand{\headrulewidth}{1pt} 
\renewcommand{\footrulewidth}{1pt}

\fancyhf{}
\lhead{\textbf{\ckey}}
\chead{\textbf{Math Article Notes}}
\rhead{\textbf{\amsclass}}
\cfoot{\articletitle}}
\rfoot{Page \thepage/\pageref{LastPage}}
\lfoot{\datenotesstarted}


% Unused in this format
% Define a command for the structure of the document title
%\newcommand{\printtitle}{
%\begin{center}
%\textbf{\Large{\doctitle}}

%\docdate

%\docauthor
%\end{center}
%}

%----------------------------------------------------------------------------------------
%	STRUCTURE MODIFICATIONS
%----------------------------------------------------------------------------------------
\usepackage[parfill]{parskip}
%\setlength{\parskip}{2pt} % Slightly increase spacing between paragraphs

% Uncomment to center section titles
%\usepackage{sectsty}
%\sectionfont{\centering}

% Uncomment for Roman numerals for section numbers
\renewcommand\thesection{\Roman{section}}
 % Input the file specifying the document layout and structure

%----------------------------------------------------------------------------
%	ARTICLE INFORMATION
%----------------------------------------------------------------------------

\articletitle{Assessing transient and persistent pain in animals} % The title of the article
\articlecitation{\cite{dubner1999assessing}} % The BibTeX citation key from your bibliography

\datenotesstarted{} % The date when these notes were first made
\docdate{\datenotesstarted; rev. \today} % The date of last revision

\docauthor{} % Author of Article
\author{K. M. Short} % Your name (this is built in now)

%----------------------------------------------------------------------------

\begin{document}

%\pagestyle{myheadings} % Use custom headers
%\markright{\doctitle} % Place the article information into the header
%----------------------------------------------------------------------------
%	PRINT ARTICLE INFORMATION (Not like that I wont)
%----------------------------------------------------------------------------

\thispagestyle{fancy} % Fancy formatting on the first page

%\printtitle % Print the title

%----------------------------------------------------------------------------
%	ARTICLE NOTES
%----------------------------------------------------------------------------

\section*{Abstract} % Unnumbered section

Lorem ipsum dolor sit amet, consectetur adipiscing elit. Nulla efficitur scelerisque eros sit amet euismod. Integer luctus, quam sed sodales lacinia, leo enim sollicitudin urna, maximus tempus nisl odio eu erat. Mauris non tristique arcu, eu venenatis nisl. Vivamus sed interdum velit. Cras ac aliquet nisl. Cras dignissim commodo dui, sed finibus nulla viverra tempor. Ut ullamcorper augue at egestas fermentum. Integer quis accumsan tellus, et efficitur dolor. Pellentesque a risus quis magna scelerisque tincidunt et quis metus. Praesent tristique suscipit ex id luctus.

%----------------------------------------------------------------------------

\section{Possible Uses of Article} % Numbered section

\begin{description}
	\item [Use Case]  Aliquam fringilla lectus vitae lorem egestas ultrices a quis nunc. Morbi consequat tincidunt ligula a mollis. In sed interdum est.
    \item [Use Case] Vivamus dolor risus, gravida et nisi nc, vehicula pharetra odio. Morbi luctus nunc ante, vitae auctor dolor luctus vitae. Sed sagittis interdum nunc et rhoncus. Curabitur rutrum gravida tellus ut dictum. Vivamus gravida nibh ante, posuere varius eros fringilla volutpat.
    \item [Use Case] In odio nisi, aliquet quis felis lobortis, commodo egestas ante. Nullam lobortis quam vel diam feugiat aliquam. Proin pellentesque congue pulvinar. Aenean congue est eu leo ultricies maximus.
\end{description}
%----------------------------------------------------------------------------
% CONCLUISION
%----------------------------------------------------------------------------

\section{Conclusion}

\begin{description}
	\item [Conclusion 1] Nulla facilisi. Sed mauris purus, imperdiet at varius porta, sagittis at nisl.
    \item [Conclusion 2] Etiam efficitur, purus eget venenatis consectetur, nunc lorem tristique enim, vitae sagittis dolor purus id mauris. Aliquam purus urna, facilisis vel mi vel, sagittis fringilla ante.
    \item [Conclusion 3] Integer tincidunt, arcu vel faucibus fringilla, orci massa dignissim lorem, finibus luctus metus nibh vulputate ex. Vivamus dui orci, mattis pretium ipsum quis, rutrum bibendum lorem.
\end{description}


%----------------------------------------------------------------------------
% RELEVANT RESULTS
%----------------------------------------------------------------------------

\section{Relevant Results}

\begin{description}
	\item [Result A] Nunc non massa eu leo sagittis aliquet. Sed commodo turpis eget est elementum, cursus cursus tortor congue. Aenean feugiat auctor tortor, vel vestibulum est feugiat et. Duis convallis volutpat cursus. 
    \item [Use of Result A] \emph{Aliquam erat volutpat. Phasellus interdum consequat condimentum.}
    \item [Result B] Morbi fermentum facilisis enim dignissim facilisis. Aenean mattis lorem sed velit gravida facilisis. In in leo nec tortor pellentesque mollis. Curabitur eget porta metus, non consectetur augue. 
    \item [Use of Result B] \emph{Fusce condimentum sit amet enim a sagittis. Lorem ipsum dolor sit amet, consectetur adipiscing elit.}
    \item [Result C] Curabitur egestas justo porttitor, commodo tellus in, consectetur dui.
    \item [Use of Result C] Aliquam erat volutpat. Phasellus interdum consequat condimentum. 
\end{description}


%----------------------------------------------------------------------------
% KEY DEFINITIONS
%----------------------------------------------------------------------------

\section{Key Definitions}

\begin{defn}
Donec ultrices odio in rhoncus rutrum. Nunc tristique venenatis nisl in aliquam. Aenean vulputate nisl quis nibh dapibus cursus. Suspendisse ornare mauris lorem, sit amet gravida massa luctus ac. 
\end{defn}


\begin{defn}
Donec ultrices odio in rhoncus rutrum. Nunc tristique venenatis nisl in aliquam. Aenean vulputate nisl quis nibh dapibus cursus. Suspendisse ornare mauris lorem, sit amet gravida massa luctus ac. 
\end{defn}



%----------------------------------------------------------------------------
% KEY THEOREMS
%----------------------------------------------------------------------------

\section{Key Theorems}

\begin{thm}
Nullam facilisis sodales erat in porttitor. Curabitur vitae leo tellus. Pellentesque fermentum, lorem id tempus blandit, massa quam condimentum dolor, et vestibulum mi eros sit amet orci. Quisque velit quam, ullamcorper eu pretium porttitor, scelerisque sit amet odio. Suspendisse quis tincidunt velit.
\end{thm}


\begin{nt}
I found their approach of subjecting helpless animals to long-term pain stimuli and monitoring depressive behaviours afterwards both novel and interesting.
\end{nt}



%----------------------------------------------------------------------------
% BIBTEX!!!
%----------------------------------------------------------------------------

\section*{Bibtex Reference}
\begin{verbatim}
	@article{ ,
    title= {},
    author= {},
    publisher= {}.
    day= {},
    month= {},
    year= {},
    journal= {},
    volume= {},
    issue= {},
    url= {},
    doi= {},
    }
\end{verbatim}


\textbf{Article Location:} \\
\url{} \\


\textbf{Author Webpage:} \\
\url{}


%----------------------------------------------------------------------------
% REMARKS
%----------------------------------------------------------------------------
\section{Remarks}

\begin{rem}
I found their approach of subjecting helpless animals to long-term pain stimuli and monitoring depressive behaviours afterwards both novel and interesting.
\end{rem}


\begin{rem}
I found their approach of subjecting helpless animals to long-term pain stimuli and monitoring depressive behaviours afterwards both novel and interesting.
\end{rem}


\begin{rem}
I found their approach of subjecting helpless animals to long-term pain stimuli and monitoring depressive behaviours afterwards both novel and interesting.
\end{rem}


%----------------------------------------------------------------------------
% References
%----------------------------------------------------------------------------


\renewcommand{\refname}{Formatted Reference}
\nocite{*}
\bibliographystyle{amsrefs}
\bibliography{sample}
%----------------------------------------------------------------------------


\listoftodos
\end{document}
