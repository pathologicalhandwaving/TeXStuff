\documentclass[10pt]{article}

\usepackage{notes}


\begin{document}

Complex numbers are great, they are not \emph{imaginary} at all, but a brilliant notation and conceptual device for simplifying and solving many different kinds of complex problems. Complex numbers allow us to solve geometric problems more easily by expressing them as trigonometric or polynomial identities which can be expanded or reduced. 

\section*{Complex Numbers}

\begin{defn}{Complex Number}

$z = x + iy = Ae^{i\theta} = A(\cos{(\theta)}) + i \sin{(\theta)}$ where $A = |z| = \sqrt{x^{2} + y^{2}}$ and $ \theta = arg(z)$

\end{defn}
 
\bigskip

\begin{defn}{Complex Conjugate}

Let $a, b, c \in \mathbb{R}$ 

The complex conjugate of $z = a + ib$ is $\bar{z} = a - ib = ce^{-i\theta}$
\end{defn}

\bigskip

\begin{nota}

$\Re{(z)} = x$ is called the real part of $z$ and $\Im{(z)} = y$ is called the imaginary part of $z$

\end{nota}

\bigskip

\begin{nota}

$\theta$ is the counter-clockwise angle between the positive $x$ axis and the vector $(x,y)$

\end{nota}

\bigskip

\begin{defn}{Primitive Root}

A primitive root of unity is the complex number $z$ represented as $z = e^{i\frac{2\pi}{n}}$ where $n \in \mathbb{Z}$. We say $z$ is a primitive $nth$ root of unity.
\end{defn}

\medskip

\begin{defn}{Roots of Unity}

The roots of unity are powers of $z - 1, z, z^{2}, \ldots, z^{n-1}$ which are all the roots of $x^{n} -1 = 0$.

\end{defn}

\medskip

\begin{defn}{Sum of Roots of Unity}

The sum of all the roots of unity, \[x^{n} = \sum^{n-1}_{k = 0} z^{n} = 0. \]

\end{defn}

\bigskip

\section*{Theorems}

\begin{fct}

$z \in \mathbb{R} \Leftrightarrow \conjugate{z} \in \mathbb{R}$

\end{fct}

\medskip

\begin{fct}

$ z \in \mathbb{C} \Leftrightarrow \conjugate{z} \in \mathbb{C} $

\end{fct}

\medskip

\begin{fct}

On the unit circle $\conjugate{z} = \frac{1}{z}$.

\end{fct}

\bigskip


\begin{thm}
Let $A, B, C, L$ be arbitrary points then the following are true:
\end{thm}

\begin{enumerate}
    \item $L$ is a point on a line passing through $C$ which is parallel to $AB \Leftrightarrow \frac{z - c}{b - a} \in \mathbb{R}$, that is $\frac{z - c}{b - a} = \frac{\conjugate{z} - \conjugate{c}}{b - a}$.
    \item $L$ is a point on a line passing through $C$ which is perpendicular to $AB \Leftrightarrow \frac{z -c}{a - b} \in \mathbb{C}$, that is $ \frac{z - c}{b - a} = - \frac{\conjugate{z} - \conjugate{c}}{b - a}$.
\end{enumerate}

\bigskip

\begin{cor}
Let $A, B$ be arbitrary points on the unit circle and let $C, L$ be arbitrary points, then the following are true:
\end{cor}

\begin{enumerate}
    \item $L$ is on the the line $AB \Leftrightarrow z + ab\conjugate{z} = a+b$.
    \item $L$ is on the line tangent to the circle at $A \Leftrightarrow z + a^{2}\conjugate{z} = 2a$.
    \item $L$ is on the line which passes through $C$ and perpendicular to $AB \Leftrightarrow z - ab\conjugate{z} = c - ab\conjugate{c}$.
    \item $L$ is on the line which passes through $C$, and is perpendicular to the tangent at $A \Leftrightarrow z - a^{2}\conjugate{z} = c - a^{2}\conjugate{c}$. 
\end{enumerate}

\section*{Complex Numbers and Trigonometry}

Complex numbers and trigonometry are tied together nicely using \emph{Euler's Formula}. 

\medskip

\begin{defn}{Euler's Formula}

$e^{i\theta} = \cos{(\theta)} + i\sin{(\theta)}$.
\end{defn}

\bigskip


\end{document}