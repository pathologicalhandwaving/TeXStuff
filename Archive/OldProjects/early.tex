\section{High School}

Mac Lane's father died when he was fifteen and his mother moved them all into their paternal grandfather's home. Mac Lane liked high school well enough and wanted to be a chemist. He stated in an interview with G. L. Alexanderson from MAA's College Mathematics Journal that he liked learning but in highschool maths had one `good' maths teacher which was in freshman algebra (who was also the football coach), the other math instructor he recalls was for the second year of geometry who did not know the subject as well as himself and that he apologetically admits he made life a bit hard for\cite{Ale1989}.

\section{Yale}

His Uncle John Mac Lane had previously sent both his own son's to Yale and in junior high school \emph{`he looked down at me and said `Saunders I will send you to Yale','} (he sent ten other nieces and nephews to college as well). So upon his graduation from high school (1926) Saunders left for Yale. He enrolled in honors chemistry and freshmen mathematics and promptly discovered he did not like laboratory work. Lester Hill, the PhD student which taught his freshman maths course encouraged him to take the Barge Prize Examination, an exam offered to freshmen each year, and upon winning he decided maths were a much better game than chemistry, he quickly switched majors. The Barge Prize accompanied a cash reward of 50\$ which Mac Lane used to purchase a pocket watch as all the male students at the time wore a vest complete with chained pocket watch. Other traditions held by the students such as the secret society's and the school paper Mac Lane did not pay any mind to. One such tradition involved sending your pants to the cleaners in order to have a nice ironed pair to wear to class each day. Mac Lane did not see the need to send out his pants to be pressed. He did however send his laundry home each week. 

Mac Lane chose the Yale College when the time came to choose between a BA and BS degree; or rather it was chosen for him since he had elected to not take Latin and Greek. He approached Professor Wallace Wilson with a request for a reading course of the \emph{Principia Mathematica}. However Wilson held the view that logic was not a part of mathematics and set him to reading \emph{Mengenleher} Felix Hausdorff's book on set theory instead.   

He tried out for track but was not a fast runner and did not make the team. Mac Lane earned the highest GPA at Yale earning him an award called a Y. Previously only awarded to athletes, Mac Lane was the first to receive the award when its reach was extended to include academic accomplishments. The Y awarded to Mac Lane gave him the opportunity to meet R. M. Hutchins at a Yale party in 1929, who asked him what he planned on doing after graduation when Mac Lane responded that he intended to attend graduate studies in Mathematics Hutchins replied \emph{`Come to Chicago, we have a great Mathematics department!'} Hutchins later sent him a letter with a fellowship offer which carried a stipend of \$1,000 dollars. One of his professors Egbert Miles had attended Chicago and was quite pleased another Oystein Ore exclaimed that Harvard or Princeton might have been better choices; though Mac Lane believes Ore would have preferred he stayed at Yale. 

\section{Chicago}

Mac Lane arrived at Chicago University for his fellowship in September of 1930, having not actually applied to the school. The Chair of Mathematics at the time Professor Gilbert A. Bliss was put off that he had not been consulted and after the usual bureaucratic soothing of ego's had commenced Mac Lane was admitted. 

By this time Mac Lane was quite sure he was interested most in foundations of mathematics. He was given a paper by Zermelo to report on by Eliakim Hastings Moore on Zermelo-Frankel set theory axioms the only alternate competitor for the foundations of mathematics to Russell and Whitehead (at that time anyway). He enthusiastically reported on the Zermelo paper to the usual sparse crowd of mathematicians and graduate students willing to sit through it. Once the room had cleared Moore sat him down and in great detail critiqued his performance and let him in on the actual meaning of the paper; Mac Lane states this talk between him and Moore was what showed him how to give mathematics talks\footnote{We can probably be safe in giving some credit to Moore on the quality of the a fore mentioned Marden lecture.}. 

Logic was then and is still in many ways thought to be separate from mathematics. He learned the Peano notation of existential quantifiers and became concerned with the concept of precise expressions of proof together with the meaning of the proof. A large influence in his work was his view that the German axiomatization of groups and rings had been performed elegantly. His masters thesis was an attempt to axiomatize exponentials as well as addition and multiplication operators. 

Mac Lane was disappointed by Chicago in part because he could see no way for him to pursue writing a thesis in logic and also by the departments lack of emphasis on writing a thesis at all. He applied for and won a fellowship to study in Germany with the goal of writing a thesis on logic. 

\todo{Perhaps fill in the chicago and yale sections a bit more?} 

