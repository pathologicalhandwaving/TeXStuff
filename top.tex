\documentclass{article}
\title{Untitled}
\usepackage{amsmath}
\usepackage{amsthm}
\usepackage{amssymb}
\usepackage[parfill]{parskip}
\theoremstyle{plain}
\newtheorem{thm}{Theorem}
\author{emi notebooks}
\date{2016-1-20}

\begin{document}
%\maketitle
\textbf{Every normed vector space has a natural topological structure:} the norm induces a metric and the metric induces a topology. 

This is a topological vector space because:

\begin{thm}
The vector addition $+ : V \times V \rightarrow V$. is jointly continuous with respect to this topology. 
\end{thm}


This follows directly from the triangle inequality obeyed by the norm.


\begin{thm}
The scalar multiplication $\cdot : K \times V \rightarrow V$, where $K$ is the underlying scalar field of $V$, is jointly continuous. 
\end{thm}


This follows from the \textbf{triangle inequality} and \textbf{homogeneity of the norm}.


Therefore, \emph{all Banach spaces and Hilbert spaces, are examples of topological vector spaces.}


There are topological vector spaces whose topology is not induced by a norm, but are still of interest in analysis. 


Examples of such spaces are spaces of \textbf{holomorphic functions on an open domain}, spaces of \textbf{infinitely differentiable functions}, the \textbf{Schwartz spaces}, and spaces of \textbf{test functions} and the spaces of distributions on them. 


These are all examples of \textbf{Montel spaces}. 


On the other hand, \emph{infinite-dimensional Montel spaces are never normable.}

\medskip

\begin{thm} 
A topological field is a topological vector space over each of its subfields.
\end{thm}

\end{document}

