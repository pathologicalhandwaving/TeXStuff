\documentclass[fronts,frame,grid,avery5388]{flashcards}

\usepackage{concmath}
\usepackage[T1]{fontenc}

\cardfrontstyle[\scshape]{plain}
\cardfrontheadstyle[\small\scshape]{center}
%\cardfronthead{Precalculus}
%\cardbackstyle{empty}
\cardfrontfootstyle[\small\bfseries]{right}
\cardfrontfoot{Precalculus}
\usepackage{amssymb}
\usepackage{amsmath}
\usepackage{datetime}

\begin{document}


\begin{flashcard}[Formula]{quadratic formula}
\vspace*{\stretch{1}}
The solutions or roots of the quadratic equation \\
$ax^2 + bx + c = 0$ are given by
\begin{equation*}
x = \dfrac{-b\pm \sqrt{b^2-4ac}}{2a}
\end{equation*}
\vspace*{\stretch{1}}
\end{flashcard}

\begin{flashcard}[Definition]{absolute value}
\vspace*{\stretch{1}}
\begin{equation*}
|x| = \left\{ \begin{array}{ll}
x & \: x \geq 0 \\
-x & \: x < 0
\end{array} \right.
\end{equation*}
\vspace*{\stretch{1}}
\end{flashcard}

\begin{flashcard}[Theorem]{properties of absolute values}
\vspace*{\stretch{1}}
\begin{enumerate}
\item $|ab| = |a||b|$
\item $\left| \dfrac{a}{b} \right| = \dfrac{|a|}{|b|}$
\item $|a+b| \leq |a| + |b|$
\item $|a-b| \geq ||a| - |b||$
\end{enumerate}

\vspace*{\stretch{1}}
\end{flashcard}


\begin{flashcard}[Definition]{equation of a line in various forms}
\vspace*{\stretch{1}}
\begin{tabular}{cc}
Form & Equation\\ \hline
\\
point--slope &  $y - y_{1} = m(x - x_{1})$\\
\\ 
slope--intercept &  $y = mx + b$\\
\\
two point &  $y - y_{1} = \dfrac{y_{2} - y_{1}}{x_{2}- x_{1}}(x - x_{1})$\\ 
\\
standard &  $Ax + By + C = 0$\\ 
\end{tabular} 
\vspace*{\stretch{1}}
\end{flashcard}

\begin{flashcard}[Definition]{equation of a circle}
\vspace*{\stretch{1}}
The equation of a circle centered at $(h,k)$
with radius $r$ is:
\begin{equation*}
(x-h)^{2} + (y-k)^{2} = r^{2}
\end{equation*}
\vspace*{\stretch{1}}
\end{flashcard}

\begin{flashcard}[Definition]{$\sin, \cos, \tan$}
\vspace*{\stretch{1}}
\setlength{\unitlength}{0.5cm}
\begin{picture}(6,3)
\thicklines
% draw 3-4-5 triangle
\put(2,0){\line(1,0){4}}
\put(6,0){\line(0,1){3}}
\put(2,0){\line(4,3){4}}
% draw box in lower right-hand corner
\put(5.6,0){\line(0,1){0.4}}
\put(6,0.4){\line(-1,0){0.4}}
% label sides of triangle
\put(6.2,1.3){opp}
\put(3.7,-0.7){adj}
\put(3.2,1.9){hyp}
\put(2.8,.1){$\theta$}
% explanatory text
\put(9,3.5){$\sin \theta = \dfrac{\text{opp}}{\text{hyp}}$}
\put(9,1.5){$\cos \theta = \dfrac{\text{adj}}{\text{hyp}}$}
\put(9,-0.5){$\tan \theta = \dfrac{\text{opp}}{\text{adj}}$}
\end{picture}
\vspace*{\stretch{1}}
\end{flashcard}

\begin{flashcard}[Definition]{$\sec, \csc, \tan, \cot$}
\vspace*{\stretch{1}}
\begin{center}
\begin{tabular}{cc}
$\sec \theta = \dfrac{1}{\cos \theta}$ &
$\csc \theta = \dfrac{1}{\sin \theta}$\\
 & \\
 & \\
$\tan \theta = \dfrac{\sin \theta}{\cos \theta}$ &
$\cot \theta = \dfrac{\cos \theta}{\sin \theta}$\\
\end{tabular} 
\end{center}
\vspace*{\stretch{1}}
\end{flashcard}

\begin{flashcard}[Definition]{midpoint formula}
\vspace*{\stretch{1}}
If $P(x_{1}, y_{1})$ and $Q(x_{2}, y_{2})$ are two points, then 
the midpoint of the line segment that joins these two points
is given by:
\begin{equation*}
\left( \dfrac{x_{1}+x_{2}}{2}, \dfrac{y_{1}+y_{2}}{2}\right) 
\end{equation*}
\vspace*{\stretch{1}}
\end{flashcard}

\begin{flashcard}[Definition]{function}
\vspace*{\stretch{1}}
A \textbf{function} is a mapping that associates with each object $x$ in one
set, which we call the \textbf{domain}, a single value $f(x)$ from a second set
which we call the \textbf{range}.
\vspace*{\stretch{1}}
\end{flashcard}

\begin{flashcard}[Definition]{even and odd functions}
\vspace*{\stretch{1}}
\begin{tabular}{ccc}
\textbf{even} & $f(-x) = f(x) \quad \text{ for all } x$ & e.g. $x^{2}, \cos(x)$\\
\\
\textbf{odd} & $f(-x) =  -f(x) \quad \text{ for all } x$ & e.g. $x, \sin(x)$\\
\end{tabular} 
\vspace*{\stretch{1}}
\end{flashcard}

\end{document}
