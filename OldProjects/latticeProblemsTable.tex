% Lattice Problems Table 2 


\documentclass{article}
\usepackage[table]{xcolor}
\usepackage[margin=1in]{geometry}
\usepackage{tabularx}
\usepackage{enumitem}

\setlist{nolistsep}
\definecolor{green}{HTML}{66FF66}
\definecolor{myGreen}{HTML}{009900}

\renewcommand{\familydefault}{\sfdefault}
\renewcommand{\arraystretch}{1.5}

\begin{document}

\begin{center}
\begin{tabularx}{\textwidth}[t]{XX}
\arrayrulecolor{darkgray}\hline
\textbf{\textcolor{myGray}{Lattice Problem Descriptions}} & \\
\hline
Shortest Vector Problem (SVP) &
\begin{minipage}[t]{\linewidth}%
\begin{itemize}
\item[Standard Definition] Given a lattice with an arbitrary basis, find the shortest vector in the lattice.
\end{itemize}
\end{minipage}\\

\arrayrulecolor{black}\hline

Variants of SVP:  &
\begin{minipage}[t]{\linewidth}%
\begin{itemize}
\item[$\alpha{SVP}$] (Approximate SVP) Given a lattice with an \\ arbitrary basis, find a vector in the lattice with length no more than an approximation factor $\alpha$, multiplied by the length of the shortest non-zero vector. \\
\item[$GapSVP_{(\beta)}$] Differentiate between cases of SVP where the solution is one or just greater than $\beta$, where $\beta$ is a function of n the number of vectors in the lattice. GapSVP decides if $\Lambda(L)$ is less than or equal to one, or if $\Lambda(L)$ is strictly greater than $\beta$. In all other cases the algorithm gives an error. \\ 
\item[$GapSVP_{(\zeta, \gamma})$] Given a basis $B$ and a number $d$ as input. We are guaranteed that all vectors under Gram-Schmidt orthogonalization have a length of at least one. So the relations $\Lambda(L(B))$ will be less than $\zeta$, also $1 \leq d \leq d \lew \zeta(n)/\gamma(n)$, \\ where $n$ is the dimension. Then the solution to $GapSVP_{(\zeta, \gamma)}$ is accepted if $\Lambda(L(B)) \geq \gamma(n).d$. For large enough values of $\zeta$, ($\zeta(n)) > 2^{n/2}$, $GapSVP_{(\zeta, \gamma)}$ is equivalent to $GapSVP_{(\gamma)}$
\item[SPIP]
\item[SG-Principal-SVP]
\end{itemize}
\end{minipage}\\

\hline

Target 1.C Halve, between 1990 and 2015, the proportion of people who suffer from hunger &
\begin{minipage}[t]{\linewidth}%
\begin{itemize}
\item[1.8] Prevalence of underweight children under five years of age
\item[1.9] Proportion of population below minimum level of dietary energy consumption
\end{itemize}
\end{minipage}\\

\arrayrulecolor{green}\hline
\textbf{\textcolor{myGreen}{Goal 2 Achieve universal primary education}} \\
\hline

Target 2.A Ensure that by 2015 children everywhere, boy and girls alike, will be able to complete a full course of primary schooling. &
\begin{minipage}[t]{\linewidth}%
\begin{itemize}
\item[2.1] Net enrollment ratio in primary education
\item[2.2] Proportion of pupils starting grade 1 who reach last grade of primary education
\item[2.3] Literacy rate of 15- to 24-year-olds, women and men
\end{itemize}
\end{minipage}\\

\hline
\multicolumn{2}{l}{%
\textbf{\textcolor{myGreen}{Goal 3 Promote gender equality and empower women}}} \\
\hline

Target 3.A Eliminate gender disparity in primary and secondary education, preferably by 2005, and in all levels of education no later than 2015 &
\begin{minipage}[t]{\linewidth}%
\begin{itemize}
\item[3.1] Ratios of girls to boys in primary, secondary and tertiary education
\item[3.2] Share of women in wage employment in the non-agricultural sector.
\end{itemize}
\end{minipage}
\end{tabularx}
\end{center}

\end{document}
