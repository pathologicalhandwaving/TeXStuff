\section{Early Years}

Leslie Saunders Mac Lane was born in Norwalk Connecticut the son of a minister's son Donald MacLane; also a minister's son. His mother Winifred was a highschool teacher. 

His father and mother grew up next door to each other. But Donald had always been shy when it came to speaking with Winifred. He was given the advice \emph{`a faint heart never won a fair lady'}, which he must have taken to heart since the two were married in 1908. On August 4th, 1909\footnote{The same year that Landau began his lectures at G\"{o}ttingen.}  Saunders Mac Lane was born as \emph{Leslie Saunders MacLane}. 

\section{What's in a Name?}

\subsection{Leslie?}

A nurse present at Mac Lane's birth had suggested the name Leslie and it had been applied. However, his mother hated the name. His mother must have hated the name quite well, as less than a month later the new family was out for a walk and his father in the dramatic style of New England Presbyterians during the period put on hand on his son, raised one hand to god, and exclaimed\cite{Mac2005} \emph{`Leslie Forget!'} Saunders never was called by a true first name ever again seeing as though both Saunders and MacLane are both traditionally surnames. 

Saunder's surname is a different story altogether; in fact it is two stories. 

 Saunders Mac Lane was previously \emph{Leslie Saunders MacLane}. His father named Donald MacLane was previously Donald MacLean. One might ask \emph{Why all this name changing? Don't most families take names rather seriously?} The answer being for the most part: \emph{Yes, they do}, and the MacLeans were no exception.

\subsection{MacLean to MacLane} 

The name changing began in Scotland as a result of the \emph{Fuadach nan G\`{a}idheal} (the eviction of the Gael) in the 18th and 19th centuries. These events are now commonly known as \emph{The Highland Clearances} and were a series of `enforced simultaneous evictions of all families living in a given area such as a glen\cite{Wat1990}.' The MacLeans departed the Straights of Mull and Castle Duarte after conceding defeat to the British in the Battle of Culloden, and settled in Ohio and Pennsylvania. 

Donald MacLane and his brother's changed the spelling of the family name to MacLane to avoid being mistakenly taken as coming from Irish descent. The distaste of being mistaken as Irish was common at the time due to anti-Irish sentiment; which at the time was quite prevalent in the North-eastern United States \footnote{See Back Matter concerning Irish Diaspora}. Being mistaken for an Irishman had the chance of significantly limiting Mac Lane and his brother's chances for employment, and various other opportunities; at least enough to make the spelling change a favorable option. So change they did\cite{Mac2005}.

\pagebreak

\begin{figure}{H}
\begin{center}
\includegraphics[scale=.5]{NINA-nyt.JPG} 
\end{center}
\caption{New York Times Article with the phrase `No Irish Need Apply'.}
\end{figure}
\cite{NYT1900}


\subsection{MacLane to Mac Lane}

The second spelling change is far less serious but perhaps was more easily managed since the first change from MacLean to Mac Lane had occurred. Mac Lane's first wife apparently found it difficult to write MacLane without inserting a space between Mac and Lane\cite{Mac2005}. So Saunders officially changed the spelling once again to suit the tendency of his wife and became Saunders Mac Lane. 