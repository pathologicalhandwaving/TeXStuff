\documentclass[10pt]{article}

% Document Information
\author{K. M. Short}
\title{Math495X: Homework 2}
\date{\today}

% Basic Packages
\usepackage{fullpage}
\usepackage{hyperref}


\usepackage[parfill]{parskip}
\usepackage{tikz}
\usepackage{amsmath}
\usepackage{amsthm}
\usepackage{amssymb}
\usepackage{stmaryrd}

% A "fancy" package
\usepackage{fancyhdr}
\pagestyle{fancy}
\usepackage{lastpage}
\setlength{\headheight}{25.2pt}
\setlength{\headsep}{0.4in}
\pagestyle{fancy}
\fancyhf{}
\rhead{\textbf{Math495X: Homework 2}}
\lhead{\textbf{K. M. Short}}
\rfoot{Page \thepage/\pageref{LastPage}}

% Envs made up in theorem, definition, and remark style (most not usually used)
\theoremstyle{plain}
\newtheorem{thm}{Theorem}
\newtheorem{lem}{Lemma}
\newtheorem{cor}{Corollary}
\newtheorem{prop}{Proposition}
\newtheorem{con}{Conjecture}
\newtheorem{crit}{Criterion}
\newtheorem{ass}{Assertion}

\theoremstyle{definition}
\newtheorem{defn}{Definition}
\newtheorem{exmp}{Example}
\newtheorem{xca}[exmp]{Exercise}
\newtheorem{cond}{Condition}
\newtheorem{prob}{Problem}
\newtheorem*{sol}{Solution}
\newtheorem*{asl}{Alternate Solution}
\newtheorem{algo}{Algorithm}
\newtheorem{que}{Question}
\newtheorem{ans}{Answer}
\newtheorem{axi}{Axiom}
\newtheorem{prot}{Property}
\newtheorem{asu}{Assumption}
\newtheorem{hyp}{Hypothesis}

\theoremstyle{remark}
\newtheorem{rem}{Remark}
\newtheorem*{nt}{Note}
\newtheorem*{nota}{Notation}
\newtheorem{cla}{Claim}
\newtheorem{summ}{Summary}
\newtheorem{case}{Case}
\newtheorem{cln}{Conclusion}

% one problem per page

%%%%%%%%%%%%%%%%%%%%%%%%%%%%%%%%%%%%
\begin{document}

\begin{prob}
\textbf{\textit{Use al-Khwarizmi's geometric method of solving quadratic equations to solve:}} \[x^2+ x = 6.\]
\end{prob}

\medskip 

\begin{sol}

\end{sol}

\medskip 

\begin{description}
\item Draw a square of size $x^2$ where the sides have length $x$.
\item The coefficients of the squared term and its root are both one so we divide one by four. 
\item On each side of the square draw a rectangle with two sides of length $x$ and two sides of length one fourth.
\item In order to \textit{‘complete the square’} four squares will need to be added to the diagram, one in each corner with both sides equal to $\frac{1}{4}$.
\item multiplying $\frac{1}{4}$ by itself gives us  $(\frac{1}{4})^2 = \frac{1}{16}$. That is, each added square has an area of $ \frac{1}{16}$.
\item We add four of these squares $4( \frac{1}{16}) = \frac{1}{4}$.
\item The length of the sides of the new extended square is then $x + \frac{1}{4}$.
\item $(x+\frac{1}{4})^2 = 6+\frac{1}{4} = \frac{25}{4}$.
\item Take the square root of both sides to get $x+\frac{1}{4} = \pm \frac{5}{2}$.
\item $x = \frac{5}{2} - \frac{1}{4} = 2$,
\item $x = - \frac{5}{2} - \frac{1}{4} = (-3)$.
\end{description}

\medskip

\begin{figure}[ht]
\begin{center}
\includegraphics[scale=.25]{m496H2.png}
\end{center}
\caption{Box with side lengths $x$}
\end{figure}

\smallskip

\begin{figure}[ht]
\begin{center}
\includegraphics[scale=.25]{m495H22.png}
\end{center}
\caption{Box with extensions added to sides.}
\end{figure}

\smallskip
\begin{figure}[ht]
\begin{center}
\includegraphics[scale=.25]{m495H23.png}
\end{center}
\caption{Complete extended box.}
\end{figure}


\pagebreak


\begin{prob}
Consider the cubic equation$x^3 + 3x - 4 = 0$. \\
\smallskip
Explain (using modern methods such as calculus) why there is only one real solution to this equation. \\
Find that solution by inspection. \\
Then write down the result of using Cardano's formula on this equation. \\
The ugly expression you get must, therefore, be equal to the nice, simple number you found by inspection. \\
Use a calculator to verify (subject to round-off error, of course) that they are in fact equal. \\
(If you can think of some clever algebraic manipulation that allows you to prove they are equal, by all means let me know. Off the top of my head, I can't.) 
\end{prob}

\medskip

\begin{sol}
To simplify how I am going to talk about the equation: \\
let $f(x) = x^3 + 3x - 4 = 0$ \\
\smallskip

First of all $f$ is an odd polynomial, so it will have two ends pointing in opposite directions, by IVT (algebraic or topological) or just plain observation this guarantees that there will be at least one real root when the function is odd. That is, there will exist some point where $f(x_0) = 0$ for some specific but arbitrarily chosen $x_0$. 

\medskip

\textit{Proof left as an exercise to the grader. To prove to yourself this is true you will get the most intuitive proof using a order topology or you could just go skip rocks over a lake. The latter will be more fun and possibly more convincing, although far less rigorous.}

\medskip 

\textbf{Why will it have exactly one real root?} \\
If we take the discriminant we get $\Delta = -540$, this fits the criteria for exactly one real root and two complex roots for $f$. 
However, Cardano wouldn't have liked this at all since two of the roots were complex. \\

\textbf{Find the first real root by inspection} \\
No need, just recall that you can factor one out of anything and check to see if that is in fact a solution when $f = 0$. 
\[ f(1) = 1^3 +3(1) - 4 = 1+ 3 -4 = 4 - 4 = 0 \checkmark \]

It works so our real root is one: $(x - 1)(x^2 +x +4)$. 

\textbf{Use Cardano's to find the remaining roots} \\

Let $x = u + v$ and make the substitutions: 
\[( (u+v)-1)((u+v)^2 + (u + v) -4 = 0\]

\end{sol}

\pagebreak

\begin{prob}
Three cards (marked 1, 2 and 3, respectively) are in a box. \\
You select one randomly, look at the number, then put it back in the box, mix the cards up completely, and then randomly draw another card, again looking at the number after you do so. \\
So, the sample space can be viewed as the set of ordered pairs ($x , y$) where both $x$ and $y$ are one of the numbers 1, 2 or 3. \\
It is of course entirely possible for you to draw the same card both times. \\
Let $A$ be the event \textit{"the first card is 1"}, $B$ the event \textit{"the sum is 3"}, and $C$ the event \textit{"the sum is 4"}. \\
Are $A$ and $C$ independent events? \\
What about $A$ and $B$? \\
Show this precisely, but can you also give an intuitive explanation for this answer? \\
\end{prob}

\medskip

\begin{sol}

\end{sol}
\end{document}